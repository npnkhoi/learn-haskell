\documentclass{article}
\begin{document}
\section{Object and Classes (WIP)}

The closest thing to an object in Haskell is a typeclass. Typeclasses are like Java interfaces -- they ensure a variable have certain properties. Examples are: `Num`, `Order` and `Read` typeclasses.

You may ask: how does Haskell glue data and methods together like in OOP? That can be achieve via `let` and `where` bindings inside functions and lists, but it is still limited in OOP sense. That is reasonable -- Haskell is purely functional, and it has its own way to achieve the tasks without using OOP concepts.

\subsection{References}
- https://stackoverflow.com/questions/5414323/does-haskell-support-object-oriented-programming
\end{document}